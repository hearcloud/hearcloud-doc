\chapter{Planificación y análisis}
\label{cap:panificacion-analisis}

\section{Planificación}
\label{sec:planificacion}

Tras definir los objetivos a completar en el transcurso de este proyecto, pasamos a presentar la planificación temporal propuesta, que será dividida en diferentes fases y presentada, finalmente, mediante un diagrama de \textit{Gantt}.

\subsection{Fases}

A continuación se describen cada una de las fases propuestas así como una serie de actividades previstas por cada una de ellas.

\begin{itemize}
	\item \textbf{Fase 1}. Especificación del proyecto. Consiste en establecer los objetivos a cumplir para completar el proyecto, así como la búsqueda de información necesaria para llevar ello a cabo.
	\begin{itemize}
		\item Búsqueda de información.
		\item Determinación de objetivos.
		\item \underline{\textit{Estimación}}: X horas
	\end{itemize}
	
	\item \textbf{Fase 2}. Planificación. Elaboración y desarrollo de la documentación correspondiente a la planificación del proyecto.
	\begin{itemize}
		\item Listar las fases del proyecto.
		\item Especificar las actividades de las que constará cada fase.
		\item Análisis de presupuesto.
		\item Temporización
		\item \underline{\textit{Estimación}}: X horas
	\end{itemize}
	
	\item \textbf{Fase 3}. Análisis y diseño. Estudio del problema a abordar, análisis de requisitos y diagramas asociados de cara a concretar la forma en que desarrollarlo.
	\begin{itemize}
		\item Análisis de requisitos
		\item Diseño de diagramas
		\item \underline{\textit{Estimación}}: X horas
	\end{itemize}
	
	\item \textbf{Fase 4}. Implementación. Tras las fases de planificación y diseño, se pasa a codificar todo lo necesario con el fin de cumplir los \hyperref[cap:objetivos]{objetivos propuestos}).
	\begin{itemize}
		\item Selección de herramientas.
		\item Desarrollo \textit{Backend}: modelos, \textit{urls}, controladores.
		\item Desarrollo \textit{Frontend}: vistas, \textit{templates}, \textit{HTML}, \textit{CSS}, \textit{Javascript}.
		\item Tareas de automatización (\textit{DevOps}).
		\item \underline{\textit{Estimación}}: X horas
	\end{itemize}
	
	\item \textbf{Fase 5}. Pruebas. Se hará uso de \textit{tests unitarios}, entre otros procedimientos, para probar el código generado y que el sistema funciona correctamente.
	\begin{itemize}
		\item Pruebas del software: modelos, vistas, formularios.
 \underline{\textit{Estimación}}: X horas
	\end{itemize}
	
	\item \textbf{Fase 6}. Documentación final. Completar la documentación restante asociada al proyecto, como manuales de usuario o anexos.
	\begin{itemize}
		\item Documentación del sistema.
		\item Manual de usuario.
		\item Documentación del proyecto.
		\item \underline{\textit{Estimación}}: X horas
	\end{itemize}
\end{itemize}

\subsection{Recursos humanos}

Se contempla la participación de representantes del público objetivo destinatario del producto con el fin de recabar información relativa a su utilidad, así como posibles deficiencias, beneficios y/o mejoras.

\subsection{Presupuesto}

La principal ventaja del uso de software libre es que no es necesario adquirir licencias de pago para usar dicho software. Como las herramientas que se van a utilizar y todo el código que se genere será libre, el coste económico y, por tanto, el presupuesto inicial del proyecto será cero.\\

El único aspecto económico relevante que habrá que considerar en este caso será, entonces, el de los posibles gastos por el uso de un servidor en el que se implante el software.

\subsection{Temporización}

Por último, para poder apreciar de una forma algo más visual la planificación temporal descrita, generamos un diagrama de \textit{Gantt} que represente dicha información.

\section{Estudio de mercado}

La creación de un producto software nuevo no es una tarea sencilla, sobre todo cuando no cuentas con una mínima información acerca del mercado en el que esperas que se desenvuelva. Por eso, realizar un estudio acerca de las alternativas ya existentes, nos ayudará a obtener datos acerca de nuestros ``competidores'' y localizar posibles enfoques promocionales de cara a resaltar las ventajas que nuestra solución pueda ofrecer frente a las suyas. Entre ellos, destacamos:

\subsection{Spotify \cite{Spotify}}
Spotify es un servicio de reproducción de música vía streaming que permite escuchar un amplio catálogo en modo radio o bien buscando por artista, álbum o listas de reproducción. Si embargo, éste es ofrecido por la plataforma y no es posible almacenar tu própia música en ella para combinarlo, de alguna manera, con lo que ya ofrecen ellos. Por tanto, si una canción no se encuentra entre sus bases de datos, tendrás que recurrir a otra alternativa para poder oirla. No obstante, si no es el caso, la plataforma representa una gran alternativa para disponer de tu selección musical en la nube y organizada según tus propios criterios en modo de listas de reproducción.\\

El servicio puede utilizarse de dos formas diferentes: la versión gratuita, con la publicidad ocasional mencionada y la limitación de no poder escuchar canciones bajo demanda en la versión móvil, y la versión ``\textit{premium}'', que elimina todas estas limitaciones, ofrece una calidad de audio mejor y la posibilidad de descargar la música para escucharla en modo sin conexión.

\subsection{Soundcloud \cite{Soundcloud}}
Soundcloud es una plataforma de distribución de audio en línea que permite a sus usuarios subir, grabar, promocionar y compartir su música de autoría original. Está, por tanto, orientada sobre todo a músicos y sellos discográficos, aunque cualquier usuario registrado tiene acceso a la música de estos y puede almacenarla en listas de reproducción, públicas o privadas, en su propio perfil para su reproducción.\\

La plataforma ofrece servicios premium en dos modalidades diferentes: ``\textit{Pro}'', que permite subir hasta seis horas de audio y, estadísticas ampliadas sobre los usuarios que reproducen tu música o deshabilitar comentarios y ``\textit{Pro Unlimited}'', que permite subir horas ilimitadas de música al usuario.

\subsection{Amazon Cloud Player \cite{ACP}} 
Amazon Cloud Player es un servicio integrado con la plataforma Amazon Music que permite al usuario almacenar y reproducir su música desde el navegador, aplicaciones móviles y de escritorio, Sonos y otras plataformas como televisiones inteligentes (se permiten hasta 10 dispositivos, que pueden autorizarse y desautorizarse desde la interfaz web). La capacidad de almacenamiento está limitada a 5GB y 250 canciones de forma gratuita, sin contar la música comprada a través de Amazon Music. Con la suscripción a Amazon Music, se puede ampliar el número hasta 250.000. Además, se permiten editar algunos metadatos de dichos ficheros, como el título, artista o álbum. La única limitación del servicio es que, aún para usar la cuenta gratuita, es necesario introducir un número de tarjeta de crédito en el sistema.

\subsection{Google Play Music \cite{GPM}}
Google Play Music es un servicio de almacenamiento y sincronización de música en la nube, a la vez que tienda musical en línea que forma parte de Google Play. Éste permite el almacenamiento gratuito de canciones propias hasta un total de 50.000 que, al sincronizarse en la nube, están disponibles desde cualquier punto. La suscripción ''\textit{all access}'' permite la reproducción en streaming bajo demanda de cualquier canción del catálogo de la tienda. Con la aplicación móvil, podremos almacenar la música de forma local para poder escucharla en modo \textit{offline}. No obstante, los usuarios deberán verificar su ubicación aportando un número de tarjeta de crédito válido a su cuenta de Google antes de poder ingresar por primera vez a la plataforma.\\

Durante el proceso de subida, el sistema intentará encontrar alguna coincidencia con el catálogo de Google Play y, si se encuenta alguno, la canción se añadirá a la librería del usuario sin la necesidad de que esta se complete de subir. Además, el servicio permite la creación de listas de reproducción automática, característica conocida como ``\textit{Instant Mix}, y recibir recomendaciones personalizadas en base a lo que más reproduzcas.

\subsection{My Music Cloud \cite{MMC}}
My Music Cloud es una plataforma que permite almacenar la música de sus usuarios en todos sus dispositivos y que dispone de un reproductor personalizado e inteligente. La aplicación sincroniza la música del usuario desde cualquier ordenador, \textit{smart phone}, tablet o televición y, una vez subida, permite reproducirla, incluso sin conexión. La versión gratuita permite subir cuanta música se quiera, aunque solamente podrán sincronizarse con los diferentes dispositivos 250. Al igual que los servicios anteriores, también dispone de una tienda online, aunque mucho más limitada en cuanto a catálogo ofertado. No obstante, la música subida por el usuario no está accesible para otros. Permite editar los metadatos básicos de cada fichero. Los formatos soportados son: aac, mp3, ra, wav, wma, m4a, ogg, 3g2, 3gp, 3gpp and 3gpp2, flac, siempre y cuando cada fichero ocupe menos de 100MB. La opción de pago amplía el numero máximo de dispositivos a sincronizar de 5 a 10, permite sincronizar y reproducir pistas ilimitadamente y elimina los anuncios.