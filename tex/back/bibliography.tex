\addcontentsline{toc}{chapter}{Bibliografía}
\label{cap:bibliografia}

% ********************************************************************
% Bibliografía general
% ********************************************************************
\subsubsection*{Articulos y referencias generales consultados durante el desarrollo proyecto:}

% Comparación de tags entre diferentes formatos
\bibitem{TMMT} Tag Mapping - Mapping Tables \url{http://wiki.hydrogenaud.io/index.php?title=Tag_Mapping}

% Audio file metadata
\bibitem{PUMID3H} Python Useful Modules. ID3 Handling \url{https://wiki.python.org/moin/UsefulModules#ID3_Handling}

% Formularios con Ajax
\bibitem{DaAFS} Django and AJAX Form Submissions - Say ``Goodbye'' to the Page Refresh \url{https://realpython.com/blog/python/django-and-ajax-form-submissions/}

% Comparación con otros servicios
\bibitem{MMC} My Music Cloud. \url{https://www.mymusiccloud.com}
\bibitem{ACP} Amazon Cloud Player. \url{https://www.music.amazon.com}
\bibitem{GPM} Google Play Music. \url{https://www.play.google.com/music}

% Aspectos legales
\bibitem{GMELNEL} ``Guardar música en la nube es legal''. Laia Reventós, 24/08/2011. \url{http://elpais.com/diario/2011/08/24/radiotv/1314136802_850215.html}


A continuación se presenta la bibliografía consultada en cada uno de los apartados de este proyecto.

\bigskip

% ********************************************************************
% Introducción
% ********************************************************************
\subsubsection*{Introducción.}

\bibitem{HDRDS} Historia del registro del sonido. \url{https://es.wikipedia.org/wiki/Historia_del_registro_del_sonido}

\bibitem{SCYUYOM} Spotify: Can you upload your own music? \url{https://community.spotify.com/t5/Desktop-Linux-Windows-Web-Player/Can-you-upload-your-own-music/m-p/41511}

% ********************************************************************
% Objetivos
% ********************************************************************
\subsubsection*{Objetivos.}

% ********************************************************************
% Analisis
% ********************************************************************
\subsubsection*{Analisis.}

% ********************************************************************
% Planificación
% ********************************************************************
\subsubsection*{Planificación.}

% ********************************************************************
% Diseño
% ********************************************************************
\subsubsection*{Diseño.}

% ********************************************************************
% Implementación
% ********************************************************************
\subsubsection*{Diseño.}

% ********************************************************************
% Pruebas
% ********************************************************************
\subsubsection*{Pruebas.}

% ********************************************************************
% Pruebas
% ********************************************************************
\subsubsection*{Conclusiones.}

% ********************************************************************
% APIs
% ********************************************************************
\subsubsection*{Páginas de consulta sobre licencias y APIs del software utilizado:}
\bibitem{CC} Creative Commons. \url{http://creativecommons.org/licenses/}
\bibitem{Dj} Django. \url{https://www.djangoproject.com/}
\bibitem{J2} Jinja2. \url{https://www.djangoproject.com/}

\bibitem{TCI} {\tt Travis CI}. \url{http://docs.travis-ci.com/}

\bigskip


% ********************************************************************
% Otro material
% ********************************************************************
\subsubsection*{Otro material}
\begin{itemize}
	\item Diversas consultas puntuales al sitio {\tt Stack OverFlow}.
	\item Material docente de las asignaturas \textbf{Fundamentos de Ingeniería del Software}, \textbf{Desarrollo de Aplicaciones para Internet} e \textbf{Infraestructura Virtual} impartidas en el \textbf{Grado en Ingeniería Informática} en la \textbf{Universidad de Granada}.
\end{itemize}
