\chapter{Conclusiones y trabajos futuros}
\label{cap:conclusiones}

\section{Conclusiones}

Tras finalizar el proyecto, hemos obtenido un \textit{prototipo funcional} de la aplicación que habíamos especificado en un principio, así como una memoria correctamente documentada sobre el desarrollo del mismo.

En base a las primeras pruebas de despliegue, el prototipo desarrollado presenta un conjunto de tecnologías e infraestructura viables para resolver el problema planteado, permitiendo así un desarrollo posterior.

Hearcloud concluye como un sistema \textit{open source} para la gestión de la bilbioteca musical de sus usuarios, resultado de 7 meses de desarrollo listo para su implantación en cualquier arquitectura.

\subsection{Conocimientos adquiridos}
Gracias al análisis y desarrollo de este proyecto, se han ido adquiriendo conocimientos en diversos ámbitos del desarrollo de software, sobre todo, en el ámbito de las tecnologías web, entre las que destacamos:

\begin{itemize}
	\item Profundización en el conocimiento del lenguaje de programación Python y librerías asociadas al tratamiento de audio, así como del framework para desarrollo web Django.
	\item Técnicas para el desarrollo de interfaces de aplicaciones \textit{frontend} mediante tecnologías web modernas como \textit{Bootstrap}, \textit{jQuery} y \textit{AJAX}.
	\item Habilidades para la gestión de proyectos software de tamaño medio aplicando técnicas de desarrollo viables.
	\item Desarrollo de pruebas automatizadas que comprueben la calidad del software generado
	\item Técnicas \textit{DevOps} para la aplicación de un flujo de desarrollo con testeo y despligue automáticos mediante integración continua.
\end{itemize} 

\subsection{Materias cursadas relacionadas con el proyecto}

El proyecto aplica conocimientos adquiridos en diversas asignaturas relacionadas con la titulación, tanto generales como de la especialidad en Tecnologías de la Información.

\begin{itemize}
	\item Materias relacionadas con los fundamentos de programación \cite{FP} \cite{MP} y Programación y Diseño Orientado a Objetos \cite{PDOO}.
	\item Fundamentos de Bases de Datos \cite{FBD}, Estructura de Datos \cite{ED}, Diseño y Desarrollo de Sistemas de Información \cite{DDSI} y Fundamentos de Ingeniería del Software \cite{FIS}.
	\item Tecnologías de la Información: Desarrollo de Aplicaciones para Internet \cite{DAI} e Infraestructura Virtual \cite{IV}.
\end{itemize}

\section{Trabajos futuros}
Si bien el prototipo resuelve los problemas iniciales planteados, aún dista de ser un software listo para producción competente de acuerdo a las exigencias del mercado actuales, por lo que se abren diferentes líneas de trabajo orientadas al desarrollo de un producto completo.

\begin{itemize}
	\item Mejoras en cuanto a la eficiencia de la interfaz de usuario en el lado del cliente.
	\item Mejoras en las funcionalidades existentes e inclusión de otras nuevas (envío de emails, soporte para nuevos formatos de audio, etc).
	\item Creación de clientes móviles y de escritorio en las diferentes plataformas.
\end{itemize}

Teniendo en cuenta todo esto, se realiza una estimación preliminar para un tiempo de desarrollo adicional de seis meses con un equipo de desarrollo de entre cuatro a seis desarrolladores para obtener dicha versión completa del producto.

\subsection{Angular}
Angular \cite{Angular} es un framework Javascript de código abierto, mantenido por Google, que se utiliza para crear lo que se denomina páginas web de una sola página \cite{SPA}. Aunque su versión 2.0 se encuentra aún en fase de pruebas frente a la actual 1.5.8, está encaminada a convertirse en todo un estándar y una tecnología muy a tener en cuenta para abordar desarrollos del lado del cliente.

Se plantea la posibilidad de portar el \textit{frontend} desarrollado durante el transcurso de este proyecto a una aplicación Angular, apostando por su versión 2 que incluye diversos cambios y mejoras, entre los que destacamos:

\begin{itemize}
	\item Enfocado a Web Components \cite{WC}. Se trata de un conjunto de especificaciones que indican cómo deben de desarrollarse webs a día de hoy.
	\item \textit{Mobile First Design}. Representa mejoras de rendimiento en terminales móviles.
	\item Detección de cambios en las vistas optimizada.
	\item Inyección de dependencias reescrita completamente para mejorar la gestión de memoria y la seguridad en el acceso a las distintas instancias que se solicitan
	\item Apuesta por TypeScript \cite{Typescript}, que es una extensión de Javascript que cuenta con interesantes características y mejoras:
	\begin{itemize}
		\item Posee elementos de programación orientada a objetos, como tipado de datos, clases, etc.
		\item Es totalmente compatible con código Javascript al tratarse de una extensión de este, por lo que pueden combinarse ambos en un mismo documento.
		\item Al no estar soportado de manera nativa por los navegadores, éste se compila a ECMAScript \cite{ECMAS} para poder ejecutarse en cualquiera de ellos.
		\item Desarrollado por Anders Hejlsberg \cite{AH} de Microsoft, padre de Turbo Pascal \cite{TurboP}, Delphi \cite{Delphi} y C\# \cite{CSharp}, lo que supone una gran confianza en él.
	\end{itemize}
\end{itemize}

\subsection{Aplicación de escritorio y dispositivos móviles}

Otra de las ventajas que ofrece Angular es que es fácilmente portable a otras plataformas. Gracias a frameworks como Electron \cite{Electron} e Ionic \cite{Ionic} sería viable el desarrollo de aplicaciones de escritorio y móviles nativas sin demasiado esfuerzo adicional por parte del equipo de desarrollo.
