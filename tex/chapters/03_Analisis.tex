\chapter{Análisis}
\label{cap:analisis}

\section{Estudio de mercado}

La creación de un producto software nuevo no es una tarea sencilla, sobre todo cuando no cuentas con una mínima información acerca del mercado en el que esperas que se desenvuelva. Por eso, realizar un estudio acerca de las alternativas ya existentes, nos ayudará a obtener datos acerca de nuestros ``competidores'' y localizar posibles enfoques promocionales de cara a resaltar las ventajas que nuestra solución pueda ofrecer frente a las suyas. Entre ellos, destacamos:

\subsection{Spotify}
Spotify es un servicio de reproducción de música vía streaming que permite escuchar un amplio catálogo en modo radio o bien buscando por artista, álbum o listas de reproducción. Si embargo, éste es ofrecido por la plataforma y no es posible almacenar tu própia música en ella para combinarlo, de alguna manera, con lo que ya ofrecen ellos. Por tanto, si una canción no se encuentra entre sus bases de datos, tendrás que recurrir a otra alternativa para poder oirla. No obstante, si no es el caso, la plataforma representa una gran alternativa para disponer de tu selección musical en la nube y organizada según tus propios criterios en modo de listas de reproducción.\\

El servicio puede utilizarse de dos formas diferentes: la versión gratuita, con la publicidad ocasional mencionada y la limitación de no poder escuchar canciones bajo demanda en la versión móvil, y la versión ``\textitt{premium}'', que elimina todas estas limitaciones, ofrece una calidad de audio mejor y la posibilidad de descargar la música para escucharla en modo sin conexión.

\subsection{Soundcloud}
Soundcloud es una plataforma de distribución de audio en línea que permite a sus usuarios subir, grabar, promocionar y compartir su música de autoría original. Está, por tanto, orientada sobre todo a músicos y sellos discográficos, aunque cualquier usuario registrado tiene acceso a la música de estos y puede almacenarla en listas de reproducción, públicas o privadas, en su propio perfil para su reproducción.\\

La plataforma ofrece servicios premium en dos modalidades diferentes: ``\textitt{Pro}'', que permite subir hasta seis horas de audio y, estadísticas ampliadas sobre los usuarios que reproducen tu música o desabilitar comentarios y ``\textitt{Pro Unlimited}'', que permite subir horas ilimitadas de música al usuario.

\subsection{Amazon Cloud Player \cite{ACP}} 
Amazon Cloud Player es un servicio integrado con la plataforma Amazon Music que permite al usuario almacenar y reproducir su música desde el navegador, aplicaciones móviles y de escritorio, Sonos y otras plataformas como televisiones inteligentes (se permiten hasta 10 dispositivos, que pueden autorizarse y desautorizarse desde la interfaz web). La capacidad de almacenamiento está limitada a 5GB y 250 canciones de forma gratuita, sin contar la música comprada a través de Amazon Music. Con la suscripción a Amazon Music, se puede ampliar el número hasta 250.000. Además, se permiten editar algunos metadatos de dichos ficheros, como el título, artista o album. La única limitación del servicio es que, aún para usar la cuenta gratuita, es necesario introducir un número de tarjeta de crédito en el sistema.

\subsection{Google Play Music \cite{GPM}} 
