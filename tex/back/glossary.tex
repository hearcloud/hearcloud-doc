\section{Glosario de términos}
\label{sec:glosario}

\begin{sortedlist}
    \sortitem{\textbf{Frontend}: es la interfaz de la aplicación, es la parte de la aplicación que el usuario utiliza para comunicarse con la misma.}

    \sortitem{\textbf{Backend}: es el motor de una aplicación, se encarga de realizar las funciones en segundo plano que se encargan de que la aplicación funcione.}

    \sortitem{\textbf{Cloud computing}: la computación en la nube es un paradigma que permite ofrecer servicios de computación a través de una red, que usualmente es Internet.}

    \sortitem{\textbf{Metadatos}: referido a ficheros informáticos, los metadatos son campos de texto que van incrustados en ellos y que añaden información adicional a los mismos.}

    \sortitem{\textbf{Software libre}: conjunto de software que por elección manifiesta de su autor, puede ser copiado, estudiado, modificado, utilizado libremente con cualquier fin y/o redistribuido con o sin cambios o mejoras en función de la licencia que éste le otorgue al mismo.}

    \sortitem{\textbf{Framework}: conjunto estandarizado de conceptos, prácticas y criterios para enfocar un tipo de problemática particular que sirve como referencia, para enfrentar y resolver nuevos problemas de índole similar.}

    \sortitem{\textbf{Template}: plantilla o documento base que, gracias a un sistema procesador de los mismos, se transofrma en documentos HTML completos.}

    \sortitem{\textbf{Forja}: plataforma de desarrollo colaborativo de software que se enfoca hacia la cooperación entre desarrolladores para la difusión de software y el soporte al usuario.}

    \sortitem{\textbf{Streaming}: distribución digital de multimedia a través de una red de computadoras, de manera que el usuario consume el producto (generalmente archivo de vídeo o audio) en paralelo mientras que se lo descarga.}
    
    \sortitem{\textbf{Diagrama de Gantt}: herramienta gráfica cuyo objetivo es exponer el tiempo de dedicación previsto para diferentes tareas o actividades a lo largo de un tiempo total determinado.}
    
    \sortitem{\textbf{Test unitario}: forma de comprobar el correcto funcionamiento de un módulo de código.}
    
    \sortitem{\textbf{URL}: identificador de recursos uniforme formado por una secuencia de caracteres, de acuerdo a un formato modélico y estándar, que designa recursos en una red, como Internet y cuya dirección puede apuntar a recursos variables en el tiempo.}
    
    \sortitem{\textbf{DevOps}: cultura o movimiento que se centra en la comunicación, colaboración e integración entre desarrolladores de software y profesionales de operaciones en las tecnologías de la información (IT).}
    
    \sortitem{\textbf{Beats Per Minute (BPM)}: unidad empleada para medir el ritmo musical que equivale al número de pulsaciones que caben en un minuto.}

    \sortitem{\textbf{ORM}: técnica de programación para convertir datos entre el sistema de tipos utilizado en un lenguaje de programación orientado a objetos y la utilización de una base de datos relacional como motor de persistencia.}

    \sortitem{\textbf{SQL}: lenguaje declarativo de acceso a bases de datos relacionales que permite especificar diversos tipos de operaciones en ellas.}

    \sortitem{\textbf{DOM}: interfaz de plataforma que proporciona un conjunto estándar de objetos para representar documentos HTML, XHTML y XML, un modelo estándar sobre cómo pueden combinarse dichos objetos, y una interfaz estándar para acceder a ellos y manipularlos.}
    
    \sortitem{\textbf{Fork}: en el ámbito del desarrollo de software, es la creación de un proyecto en una dirección distinta de la principal u oficial tomando el código fuente del proyecto ya existente.}

\end{sortedlist}

\newpage
