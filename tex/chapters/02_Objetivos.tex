\chapter{Objetivos}
\label{cap:objetivos}

El objetivo del proyecto es la creación de una plataforma web donde los usuarios puedan almacenar su biblioteca musical y ésta se encuentre disponible para ellos en cualquier lugar y momento que deseen a través de internet. Se pretende establecer una metodología de trabajo flexible, en la que puedan ir incluyéndose mejoras incrementalmente manteniéndose la integridad del sistema. \\

Si bien, éste es el objetivo final a alcanzar, podemos destacar una serie de objetivos principales a cumplir que darán pié al mismo:

\begin{itemize}
	\item \textbf{OBJ-1.} Estudiar el estado actual del mercado en cuanto a productos y soluciones de este tipo, analizando sus ventajas y desventajas correspondientes.

	\item \textbf{OBJ-2.} Proporcionar una metodología flexible y sencilla, pero a la vez robusta y viable que se pueda aplicar durante el desarrollo íntegro del proyecto. 

	\item \textbf{OBJ-3.} Crear un sistema flexible que pueda ser adaptado con facilidad por otros usuarios que así lo deseen, ya sea como contribución al proyecto original o para su propia implantación de manera externa.

	\item \textbf{OBJ-4.} Dotar al proyecto de la posibilidad de ser alojado en la nube, buscando una alta disponibilidad y calidad en la experiencia de usuario.

	\item \textbf{OBJ-5.} Llevar a cabo un desarrollo abierto tanto del proyecto como de su documentación (véase apéndice \hyperref[sec:licencias]{Licencias}), dando como resultado un producto \textit{open source} basado en tecnología libre.
\end{itemize}

\bigskip

\section{Alcance de los objetivos}

La plataforma resultante podrá ser implantada en cualquier servidor en el que quiera desplegarse siempre que puedan satisfacerse las dependencias del mismo. Por otra parte, al tratarse de un sistema open source y, puesto que el software es de dominio público, se podría buscar un posible nicho de mercado en la prestación del servicio en la nube.
