\chapter{Introducción}
\label{cap:introduccion}

\section{Antecedentes históricos}

En el ámbito de las prácticas culturales colectivas y aficiones individuales de cualquier sociedad, está muy asentado el gusto por la música, sea del tipo que sea, hecho por el cual resulta inevitable que, con el tiempo, hayan surgido dispositivos que permitieran registrar dichos sonidos y almacenarlos para su posterior reproducción.

La historia contemporánea del registro del sonido comienza en 1857, con el fonoautógrafo de Leon Scott, aunque en 1877, Thomas Edison crea el primer artefacto capaz de grabar y reproducir sonido, el fonógrafo, que rápidamente fue sustituido por el gramófono, debido a sus diversas ventajas con respecto a su predecesor. Más tarde, en 1940, aparece el disco de vinilo, logrando una mayor duración y calidad de sonido. Durante ese mismo año, se desarrolla el magnetófono de bobina abierta, que aporta grabaciones de larga duración y buena fidelidad, el cual tuvo un gran éxito tanto en uso privado como profesional. El casete compacto surge posteriormente basándose en esos mismos principios, seguido del microcasete, minicasete, VHS, Casete Compacto Digital y mini DV. Ya en 1979, se produce uno de los inventos más revolucionarios, el disco compacto, que supuso el primer formato digital para audio, desplazando al disco de vinilo y al casete. No obstante, la mayor innovación en este ámbito, el formato MP3, se empieza a gestar en 1986 y, en 1995 Brandenburg lo utiliza por primera vez en su propio ordenador, convirtiéndose hoy en día en el formato más popular de audio.

Cada soporte de los aquí descritos, ha llevado consigo una forma diferente de organizar los contenidos musicales por parte de los usuarios, tendiendo a una reducción y optimización del espacio necesario para albergar dicha biblioteca musical. 

Con la introducción en el mercado de los ordenadores personales y los diferentes formatos de archivos de audio digital, las alternativas disponibles para llevar a cabo esta organización han aumentado de manera exponencial y se han visto multiplicadas, posteriormente, debido a la aparición de internet.

\section{Motivación}

Uno de los principales problemas que surgen del almacenamiento de música en los ordenadores personales en la actualidad es su portabilidad, pues existe un gran abanico de posibilidades en cuanto a dispositivos de reproducción se refiere: reproductores MP3, iPods, teléfonos móviles, tabletas, autoradios, etc. 

No obstante, normalmente, el usuario debe elegir un dispositivo en el que almacenar su biblioteca y, si quisiera portarla a otro, copiarla al mismo mediante algún método de transferencia de ficheros. El principal problema de ello, es mantener la sincronización entre estos dispositivos.

Este proyecto surge como respuesta a esa necesidad de los usuarios de organizar su música y mantenerla ``sincronizada'' en todos sus dispositivos, de manera que esté disponible en el lugar que se desee.

